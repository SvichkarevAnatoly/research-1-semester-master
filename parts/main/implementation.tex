\section*{Программная реализация DEEP}
\addcontentsline{toc}{subsection}{Программная реализация DEEP}

DEEP реализован на языке C с интерфейсом в виде командной строки.
Пользователь задаёт команду с параметрами для тестирования,
которая запускает модель и выводит все измерения
разницы между решением модели
и экспериментальными данными для минимизации.
Исполнение начинается с блока инициализации,
в котором случайным образом генерируются
начальные вектора параметров индивида
длины \begin{math}n\end{math}
с ограничениями сверху и снизу
на значения координат вектора.

Для ограничений в виде неравенств
применяются преобразования к равенству
с использованием тригонометрического синуса
или гиперболического тангенса.

Текущее множество векторов параметров помещается
в блок популяции и на каждом из векторов
вычисляется целевой функционал.
Главный цикл алгоритма состоит из блоков
рекомбинации, оценивания и отбора.

Для параллельного вычисления индивидуальных решений
используется ThreadPool API из библиотеки GLIB.
Задачи на вычисления целевого функционала,
указанного пользователем,
для каждого пробного вектора
помещаются в асинхронную очередь задач
и запускаются как только будет доступен свободный поток. 
Все потоки, которые начали своё выполнение
на текущей итерации должны завершиться
до следующей итерации.
Следующее поколение генерируется в блоке отбора
из популяции индивидов в соответствии
со значением целевого функционала.

