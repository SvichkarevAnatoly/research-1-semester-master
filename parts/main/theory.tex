\section*{Теория метода DEEP}
\addcontentsline{toc}{subsection}{Теория метода DEEP}

Ниже приведено описание оригинального метода разностной эволюции
и его модификации DEEP.

Разностная эволючия (РЭ) ---
многомерный стохастический итерационный алгоритм минимизации.

Метод оперирует случайно сгенерированными векторами параметров,
называемых индивидами. 
Под вектором понимается точка n-мерного пространства
из области определения целевой функции,
которую требуется минимизировать.
Множество индивидов называется популяцией.
Одна итерация популяции РЭ называется поколением.
Новое поколение генерируется случайным образом
по заданной схеме из индивидов текущего поколения.

Идея генерации нового поколения в оригинальном алгоритме
заключается в следующем.
Новое поколение генерируется в три этапа.
Для каждого индивида текущего поколения
выбираются случайным образом 3 другие индивида
из поколения и вычисляется мутантный вектор по формуле:

TODO

Производится операция скрещивания мутантного вектора с исходным,
замещением некоторых координат значениями из исходного вектора.
Полученный вектор называется пробным вектором.
Если значение целевой функции на нём стало меньше,
чем было на исходном, то пробный вектор добавляется в новое поколение.
Если нет, то в новое поколение переходит исходный индивид.
Таким образом, в каждом следующем поколение новые индивиды
стремится уменьшить значение целевой функции
и при определённых условиях может быть найден глобальный минимум.

Опишем две модификации РЭ, разработанной в \cite{KozlovThesis}.

TODO Скрещивание с учётом значения функционала.

TODO Скрещивание для поддержания разнообразия индивидов.

