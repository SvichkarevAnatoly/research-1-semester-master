\section*{Сегментация траекторий частиц}
\addcontentsline{toc}{subsection}{Сегментация траекторий частиц}

Обратной задачей математического моделирования
может являться выявление параметров взаимодействий биологических молекул,
которые не могут быть измерены напрямую из экспериментов.

Для тестирования программы использовалась
прикладная задача трэкинга частиц
и сегментации траекторий \cite{track}.

Исходными данными являются
серии снимков биологических макромолекул,
а именно эндосом эпидермального фактора роста
в клетках HeLa.

Первым этапом обработки происходит выделение
положения частиц из исходных снимков.
Затем с помощью стороннего программного обеспечения
производится построение траекторий.
Главной задачей является выяснить
как частицы движутся в клетке,
т.е. разбить траекторию на участки,
соответствующие либо диффузии,
либо направленному движению по микротрубочке.

Для последней задачи использовались
скрытые марковские модели,
что позволило получить
параметры движения и
построить вероятностную модель
биологического процесса.

