\chapter*{ВВЕДЕНИЕ}
\addcontentsline{toc}{section}{ВВЕДЕНИЕ}

Целью работы является развитие метода ППРЭ (полностью параллельная разностная эволюция, DEEP, Differential Evolution Entirely Parallel Method) минимизации функции, для оценки параметров математической модели по экспериментальным данным.

Математические модели в биоинформатике в большинстве случаев создаются в таких компьютерных системах расчетов как MATLAB, Octave, R и др. Следует также учитывать запись моделей на языке SBML. Нахождение параметров в таких моделях требует многократного вычисления решений, что влечет большие накладные расходы на запуск того или иного интерпретатора. Однако, интерпретатор может быть встроен в программу ППРЭ, что позволит запускать нужное число копий один раз, и, тем самым сократить время вычислений, для некоторых задач в разы.

