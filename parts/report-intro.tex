\chapter*{ВВЕДЕНИЕ}
\addcontentsline{toc}{section}{ВВЕДЕНИЕ}

Целью работы является развитие метода ППРЭ (полностью параллельная разностная эволюция, DEEP, Differential Evolution Entirely Parallel Method) минимизации функции, для оценки параметров математической модели по экспериментальным данным.

Математические модели в биоинформатике в большинстве случаев создаются в таких компьютерных системах расчетов как MATLAB, Octave, R и др. Следует также учитывать запись моделей на языке SBML. Нахождение параметров в таких моделях требует многократного вычисления решений, что влечет большие накладные расходы на запуск того или иного интерпретатора. Однако, интерпретатор может быть встроен в программу ППРЭ, что позволит запускать нужное число копий один раз, и, тем самым сократить время вычислений, для некоторых задач в разы.


DEEP представляет из себя эффективный метод решения обратной задачи математического моделирования, а именно модификация глобального стохастического метода и метод оптимального наискорейшего спуска для поиска минимума функционала качества на основе необходимого условия стационарности первого порядка. Проведённые численные эксперименты на тестовой задаче показали высокую эффективность параллелизации и превосходство в скорости сходимости над другими современными методами.

В последние годы в биологии развивается новое направление - системная биология, целью которой является получение информации о биологическом объекте как о системе взаимодействующих компонент и процессов.

Другим важным классом программ являются пакеты программ для решения обратной задачи математического моделирования. Дело в том, что несмотря на огромный прогресс молекулярной биологии в последние годы, многие параметры взаимодействий биологических молекул все ещё не могут быть измерены в экспериментах и, поэтому, их приходится находить, решая обратную задачу, т.е. путем подгонки решений модельных уравнений к экспериментальным данным по экспрессии генов. Формальная постановка задачи обычно сводится к минимизации некоего функционала качества при условии удовлетворения наложенных на параметры ограничений.

К числу наиболее популярных глобальных методов минимизации относится метод Метрополиса (так называемый "численный отжиг") и такие разновидности генетических алгоритмов как эволюционные стратегии и разностная эволюция. Эти методы имеют важное преимущество, состоящее в их способности обходить локальные минимумы функционала качества. Однако их существенным недостатком является то, что для получения результатов в реальных задачах требуется весьма большое количество вычислительного времени, а используемые процедуры являются эвристическими и далеко не гарантируют нахождения "лучшего" решения. Гораздо более экономичными по времени расчета являются локальные методы, основанные на применении градиентного спуска и теории оптимального управления. Однако, для применения этих методов необходимо получить начальное приближение параметров. Таким образом, создание новых эффективных алгоритмов решения обратной задачи математического моделирования, а также ускорение работы алгоритмов путем их параллелизации для вычисления на компьютерных кластерах необходимо для успешного проведения исследований в области системной биологии.

Разработать эффективные методы решения обратной задачи математического моделирования, которые позволяют определять параметры путём подгонки результатов моделирования к количественным данным.

Разработана модификация глобального стохастического метода решения обратной задачи математического моделирования, которая использует для повышения эффективности несколько критериев качества для каждого набора параметров модели.

Разработаны эффективные методы решения обратной задачи математического моделирования, а именно модификации глобального стохастического метода и метод оптимального наискорейшего спуска для поиска минимума функционала качества на основе необходимого условия стационарности первого порядка. Проведённые численные эксперименты на тестовой задаче показали высокую эффективность параллелизации и превосходство в скорости сходимости на другими современными методами.
