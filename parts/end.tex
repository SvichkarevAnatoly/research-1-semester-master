\chapter*{ЗАКЛЮЧЕНИЕ}
\addcontentsline{toc}{section}{ЗАКЛЮЧЕНИЕ}

Целью работы было развитие метода ППРЭ,
путём улучшения интеграции
между методом оптимизации
и функционалом качества.
В результате исследования
предполагалось увеличить
эффективность реализованной
модификации метода ППРЭ
путём встраивания интерпретаторов
и экономии процессорного времени
на их инициализацию.

ППРЭ является методом
полностью параллельной разностной эволюции,
который используется для решения
обратной задачи математического моделирования.

Показана эффективность
оптимизации в сравнении с
прошлой реализацией.
Модификация ППРЭ
была разработана на основе
открытой кроссплатформенной
библиотеки GLib.
Новый метод
превзошёл предыдущую реализацию
по времени выполнения на задаче
сегментации траекторий частиц
в 4 раза, используя при запуске
4 потока и 4 интерпретатора.

Полученная реализация
была протестирована на
прикладной задаче
сегментации траекторий частиц
с изменением числа запускаемых
интерпретаторов в диапазоне
от 1 до 8.
ППРЭ использовалась в этой задаче
для нахождения вероятностей переходов и
параметров движений
скрытой марковской модели по
экспериментальным данным.
В качестве тестовой функции
также использовалась
смещённая функция Растригина.

Работа развивает метод deep
в направлении улучшения интеграции,
что ведёт к увеличению количества
проверяемых гипотез.

