\chapter*{ВВЕДЕНИЕ}
\addcontentsline{toc}{section}{ВВЕДЕНИЕ}

Проведение вычислительного эксперимента
в большинстве случаев обходится значительно дешевле,
чем проведение соответствующего эксперимента над реальным биологическим объектом.
Именно поэтому важно развивать новые подходы в моделировании.

Важным классом программ для моделирования
являются пакеты программ
для решения обратной задачи математического моделирования.
Постановка задачи ставится
как минимизация функционала качества
при условии ограничений на параметры.

Метод полностью параллельной разностной эволюции (далее ППРЭ, DEEP) \cite{Kozlov11, Kozlov13} является модификацией стохастического метода оптимизации, предложенного в \cite{Storn95}.
DEEP представляет из себя эффективный метод решения обратной задачи математического моделирования, а именно модификация глобального стохастического метода и метод оптимального наискорейшего спуска для поиска минимума функционала качества на основе необходимого условия стационарности первого порядка.

Целью работы является развитие метода ППРЭ, а именно минимизации функции, для оценки параметров математической модели по экспериментальным данным.

Математические модели в биоинформатике в большинстве случаев создаются в таких компьютерных системах расчетов как R, MATLAB, Octave и других.
Нахождение параметров в таких моделях требует многократного вычисления решений, что влечет большие накладные расходы на запуск того или иного интерпретатора.
Однако, интерпретатор может быть встроен в программу ППРЭ, что позволит запускать нужное число копий один раз, и, тем самым сократить время вычислений, для некоторых задач в разы.

Таким образом, развитие методов решения обратной задачи математического моделирования и эффективное распараллеливание существующих решений является важным для исследований системной биологии.

