\chapter*{ВВЕДЕНИЕ}
\addcontentsline{toc}{section}{ВВЕДЕНИЕ}

Одной из активно развивающихся междисциплинарных областей биологии является системная биология. 
Данная наука анализирует сложные биологические объекты с целью получения информации о системе взаимодействующих компонент и процессов.

Главным инструментом анализа в системной биологии является моделирование.
Оно нужно как для проверки гипотез при анализе экспериментальных данных, так и для предсказания поведения системы в условиях отличных от эксперимента или в таких условиях, для которых затруднительно поставить эксперимент.
Проведение вычислительного эксперимента в большинстве случаев обходится значительно дешевле, чем проведение соответствующего эксперимента над реальным биологическим объектом.
Именно поэтому важно развивать новые подходы в моделировании.

Важным классом программ для моделирования являются пакеты программ для решения обратной задачи математического моделирования.
Существуют задачи, где необходимо измерить параметры взаимодействия биологических молекул, однако современный уровень технологий не позволяет это сделать напрямую в эксперименте.
Поэтому приходится их находить, решая обратную задачу --- путём подгонки решений модельных уравнений к экспериментальных данным.
Постановка задачи ставится как минимизация функционала качества при условии ограничений на параметры.

Метод полностью параллельной разностной эволюции (далее ППРЭ, DEEP) \cite{Kozlov11, Kozlov13} является модификацией стохастического метода оптимизации, предложенного в \cite{Storn95}.
DEEP представляет из себя эффективный метод решения обратной задачи математического моделирования, а именно модификация глобального стохастического метода и метод оптимального наискорейшего спуска для поиска минимума функционала качества на основе необходимого условия стационарности первого порядка.

Целью работы является развитие метода ППРЭ, а именно минимизации функции, для оценки параметров математической модели по экспериментальным данным.

Математические модели в биоинформатике в большинстве случаев создаются в таких компьютерных системах расчетов как R, MATLAB, Octave и других.
Нахождение параметров в таких моделях требует многократного вычисления решений, что влечет большие накладные расходы на запуск того или иного интерпретатора. Однако, интерпретатор может быть встроен в программу ППРЭ, что позволит запускать нужное число копий один раз, и, тем самым сократить время вычислений, для некоторых задач в разы.

