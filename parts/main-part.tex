\setcounter{figure}{0} \setcounter{table}{0} \setcounter{equation}{0}
\chapter*{ОСНОВНАЯ ЧАСТЬ}
\addcontentsline{toc}{section}{ОСНОВНАЯ ЧАСТЬ}
\section*{Теория метода DEEP}
\addcontentsline{toc}{subsection}{Теория метода DEEP}

Выделяют два класса задач, связанные с математическими моделями: это прямые и обратные.

В прямой задаче структура и параметры модели считаются известными, необходимо получить новую полезную информацию об объекте. Примером такой задачи может быть вычисление максимальной статической нагрузки на мост.

Для обратной задачи чаще всего модель известна, но не известны некоторые параметры модели. Задача заключается в нахождение этих параметров, например, из данных уже проведённых экспериментов, либо постановки дополнительных экспериментов, называемых активным наблюдением.

Примером обратной задачи математического моделирования может являться выявление параметров взаимодействий биологических молекул, которые не могут быть измерены напрямую из экспериментов. Формальная постановка данной задачи к минимизации некоего функционала качества при соблюдение ограничений на параметры.

Методы оптимизации можно классифицировать в соответствии с задачами оптимизации на локальные методы, которые сходятся к локальному экстремуму целевой функции (в случае унимодальной функции, экстремум единственный и одновременно является глобальным экстремумом) и глобальные методы, которые стремятся к выявлению глобальных тенденций поведения целевой функции и поиску глобального экстремума.


К числу наиболее популярных глобальных методов минимизации относится метод Метрополиса (так называемый "численный отжиг") и такие разновидности генетических алгоритмов как эволюционные стратегии и разностная эволюция.
Эти методы имеют важное преимущество, состоящее в их способности обходить локальные минимумы функционала качества.
Однако их существенным недостатком является то, что для получения результатов в реальных задачах требуется весьма большое количество вычислительного времени, а используемые процедуры являются эвристическими и далеко не гарантируют нахождения "лучшего" решения.
Гораздо более экономичными по времени расчета являются локальные методы, основанные на применении градиентного спуска и теории оптимального управления.
Однако, для применения этих методов необходимо получить начальное приближение параметров.
Таким образом, создание новых эффективных алгоритмов решения обратной задачи математического моделирования, а также ускорение работы алгоритмов путем их параллелизации для вычисления на компьютерных кластерах необходимо для успешного проведения исследований в области системной биологии.

Разработать эффективные методы решения обратной задачи математического моделирования, которые позволяют определять параметры путём подгонки результатов моделирования к количественным данным.

Разработана модификация глобального стохастического метода решения обратной задачи математического моделирования, которая использует для повышения эффективности несколько критериев качества для каждого набора параметров модели.

Разработаны эффективные методы решения обратной задачи математического моделирования, а именно модификации глобального стохастического метода и метод оптимального наискорейшего спуска для поиска минимума функционала качества на основе необходимого условия стационарности первого порядка.
Проведённые численные эксперименты на тестовой задаче показали высокую эффективность параллелизации и превосходство в скорости сходимости на другими современными методами.

Во второй части этой главы дано описание оригинального метода разностной эволюции и его модификации DEEP, разработанной автором.
Описано преобразование ограничений в виде неравенств.
Приведено описание и вывод метода оптимального наискорейшего спуска.

Строится кривая сходимости метода ППРЭ для задачи реконструкции генной сети по тестовым данным, определяется параллельная эффективность и ускорение для ППРЭ, а так же проводится сравнение с методом эволюционной стратегии.
Приведены результаты численных экспериментов по применению метода оптимального наискорейшего спуска в биоинформатике и выводы из них.


\section*{Программная реализация DEEP}
\addcontentsline{toc}{subsection}{Программная реализация DEEP}
TODO

\section*{Использование интерпретаторов}
\addcontentsline{toc}{subsection}{Использование интерпретаторов}
TODO

\section*{Тестовые функции}
\addcontentsline{toc}{subsection}{Тестовые функции}
TODO

\section*{Численные эксперименты}
\addcontentsline{toc}{subsection}{Численные эксперименты}
TODO

\section*{Сегментация траекторий частиц}
\addcontentsline{toc}{subsection}{Сегментация траекторий частиц}
TODO

